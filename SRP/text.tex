\documentclass[a4paper, 10pt]{article}
\usepackage{physics}
\usepackage[english, russian]{babel}
\usepackage[utf8x]{inputenc} 
\usepackage{mathrsfs}
\usepackage{geometry}
\usepackage{graphicx}
\usepackage{pgfplots} 
\usepackage{wrapfig}
\usepackage{caption}
\pgfplotsset{ every non boxed x axis/.append style={x axis line style=-},
     every non boxed y axis/.append style={y axis line style=-}}
\geometry{
  a4paper,
  top=25mm, 
  right=15mm, 
  bottom=25mm, 
  left=30mm
}
\title{Уравнение Линдблада для двухуровневой системы, взаимодействующей с термостатом}
\author{Pan Vyacheslav Igorevich}
\date{\today}
\begin{document}
\maketitle
    \begin{abstract}
        Уравнение Шредингера (\ref{link name}), широко применяемое для нахождения волновой функции, имеет ограниченное применение, 
        так как, описывая изменение системы только под действием потенциальных сил, позволяет определить только чистые 
        состояния\footnote{Полностью известное квантовое состояние.} и не способно описать диссипатицию\footnote{Необратимая потеря энергии.}
        квантовой системы.

        \begin{equation} \label{link name}
            i\hbar\partial_{t} \ket{\psi} = \hat{H}\ket{\psi}
        \end{equation}

        В то же время матрица плотности может задавать как чистые, так и смешанные состояния. Уравнение Линдблада
        (\ref{link name}), рассматривоемое в данной работе, является уравнением матрицы плотности, описывающим ее эволюцию.

        \begin{equation}\label{link name}
            \partial_t \hat{\rho} = -\frac{i}{\hbar} [\hat{H},\hat{\rho}] + \sum_i \gamma_i ( L_{i} \rho L^{\dagger} - \frac{1}{2} [L_{i}^{\dagger} L_i , \hat{\rho}]) 
        \end{equation}

    \end{abstract}

    \section{Введение}
        Введем некоторые постулаты квантовой механики для чистых состояний.
        \\ \\
        \begin{itshape}
            \centering Постулат 1. С любой закрытой квантовой системой связано конечномерное или бесконечномерное 
            Гильбертово пространство\footnote{Линейное пространство, в котором норма порождается скалярным произведеднием.} $\mathscr{H}$ 
            над полем комплексных чисел, 
            которому принадлежит вектор состояний ($\ket{\psi} \in \mathscr{H}$).
        \end{itshape}
        \\ \\
        Состояния системы, описываемые векторами состояний называют чистыми. Зная вектор состояния системы, мы владеем наибольшей возможной 
        информацией о ней.
        Вектор состояния $\Psi$ в нотации Дирака можно записать как 
     
        \begin{equation}\label{link name}
            \Psi = \sum_i a_i \ket{\psi_i}
        \end{equation}
        где $\psi_i$ --- возможное состояние системы, $a_i$ --- амплитуда вероятности нахождения системы в состоянии с индексом i. Так как суммарная 
        вероятность всех состояний должна ровняться единице,
        
        \begin{equation}\label{link name}
            \sum_{i} a_i^2 = 1
        \end{equation}

        В случае, если мы не владеем полным представлением о состоянии системы, мы говорим, что она находится в смешанном состоянии. Как было сказанно выше,
        для описания смешанных систем используется оператор $\rho$, принадлежащий Гильбертову пространству, называемый матрицей плотности (или оператором 
        плотности) и задается как

        \begin{equation}\label{link name}
            \hat{\rho} = \sum_i p_i \ket{\psi_i} \bra{\psi_i}
        \end{equation}

        где $p_i$ является вероятностью нахождения состояния $\psi_i$, а $\ket{\psi_i}\bra{\psi_i}$ --- соответствующий оператор проекции.
        След матрицы плотности равен 1 по условию нормировки (Tr[$\rho$]=1), a сама матрица должна быть положительна, 
        по определению вероятности ($\rho > 1$). \\
        В силу утверждения (4) случае если $Tr[\rho^2] = Tr[\rho] = 1$ мы считаем состояние чистым. В случае $tr[\rho^2] < 1$ состояние смешанное.
        Матрица плотности представляет собой квадратную матрицу размерности $N \times N$, где N --- количество базисных векторов соответствующего 
        Гильбертова пространства.
        \\ \\
        \begin{itshape}
            Постулат 2.Пусть до измерения система находилась в чистом состоянии $\psi$. В результате измерения микросистема 
            переходит в одно из состояний различимых макроприборомю Согласно постулату 1, каждому такому состоянию соответствует векто $\ket{\varphi_i}$. 
            Тогда вектор состояний $\ket{\psi}$ можно записать как линейнуюю суперпозицию по набору состояний $\ket{\varphi_i}$:
        \end{itshape}
        \\

        \begin{equation}
            \ket{\psi}= c_i \sum_i \ket{\varphi_i}
        \end{equation}
        \\ 
        \begin{itshape}
            где $c_i$ - набор комплексных чисел, которые определяются с помощью скалярного произведения
        \end{itshape}
        \\

        \begin{equation}
            c_i = \bra{\varphi_i} \ket{\psi_i}
        \end{equation}

        
        \begin{itshape}
            Постулат 3. Эволюция чистых состояний закрытой квантовой системы описываетя уравнением Шредингера.
        \end{itshape}
        \\

        \begin{equation}
            \frac{d}{dt} \ket{\psi(t)} = -i \hbar \hat{H} \ket{\psi(t)}
        \end{equation}
        \\

        Если нам известно, что в точке t = 0 система находится в состоянии $\ket{\psi(0)}$, то переходя в систему единиц измерения, в которой $\hbar = 1$, формальное решение уравнения Шредингера можно представить в виде:
        \\
        \begin{equation}
            \ket{\psi(t)} = e^{-i\hat{H}t} \ket{\psi(0)}
        \end{equation}

        \begin{itshape}
            Постулат 4. Пространство состояний составной системы, состоящей из N числа подсистем, является тензорным произведением всех ее компонентов $\mathscr{H} = \mathscr{H}_1  \otimes  \mathscr{H}_2  \otimes ... \otimes \mathscr{H}_N$. Если подсистема принадлежащая Гильбертову пространству приготовлена в состоянии $\ket{\psi_i}$, 
            то ее пространство состояний будет иметь вид $\ket{\psi} = \ket{\psi_1}  \otimes  \ket{\psi_2}  \otimes  ... \otimes  \ket{\psi_N}$. В случае составной системы ее конечное смешанное состояние будет иметь вид $\rho = \rho_1  \otimes  \rho_2  \otimes ... \otimes \rho_N$. Взаимозависимые состояния зовуться запутанными.
        \end{itshape}
        \\
        \\
        Пусть существует система $\mathscr{H} = \mathscr{H}_\alpha \otimes \mathscr{H}_\beta$. Для описания состояния подсистемы $\alpha$ необходимо определить матрицу плотности соответствующую этой подсистеме $\rho_\alpha$. Для этого запишем матрицу плотности системы 
        \begin{equation}
            \rho = \left[\sum_{i,j,k,l} \ket{\alpha_i}\bra{\alpha_j} \otimes \ket{\beta_k} \bra{\beta_l}\right]
        \end{equation}
        Заметим, что если взять след матрицы по базису
        \newpage
        \section{Описание эволюции двухуровневой системы во времени}

    \begin{wrapfigure}{r}{0.5\textwidth}
        \begin{tikzpicture}
            \begin{axis}[
                width = 0.5\textwidth,
                xmin = 0, xmax = 12, ymin = 0, ymax = 1, 
                axis lines = middle,
                anchor=north west,
                every axis plot/.append style={ultra thick},
                axis line style={shorten >=-15pt, shorten <=-15pt},
                xlabel = время, ylabel = вероятность,
                x label style={at={(axis description cs:0.5,-0.1)},anchor=north},
                y label style={at={(axis description cs:-0.1,.5)},rotate=90,anchor=south}]
                \addplot[color = cyan, samples = 1000, domain=0:12]{0.4*(sin(deg(x - pi/2)) + 1)};
                \addplot[color = orange, samples = 1000, domain=0:12]{0.4*cos(deg(x)) + 0.6};
            \end{axis}
        \end{tikzpicture}
        \captionsetup{justification= centering, margin = 0.1 cm}
        \caption{Распределение вероятности между двумя кубитами}
    \end{wrapfigure}
    Пусть $\ket{0}$ и $\ket{1}$ являются собственными состояниями Гамильтониана  $\hat{H}$. Тогда эволюция изолированной системы состоящей из одного кубита будет описываться 
    \\
    \begin{equation}
        \hat{H} = E_0 \ket{0}\bra{0} + E_1 \ket{1}\bra{1}
    \end{equation}
   Если кубит в момент времени t = 0 находился в состоянии $\ket{\psi(0)} = \ket{1}$, состояние в произвольный момент времени t будет соответствовать уравнению $\ket{\psi(t)}=e^{-i\hat{H}t}\ket{1} = e^{-iE_1t}\ket{1}$. 
   Эволюцию системы в таком случае можно описать добавлением фазы к состоянию получая
    \begin{equation}
        \hat{H} = E \ket{1}\bra{1}, 
    \end{equation}
    Где, $E \equiv E_1$
    \end{document}

    

